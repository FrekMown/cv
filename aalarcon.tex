\documentclass[9pt,a4paper]{altacv}%
\usepackage[hidelinks]{hyperref}

%% AltaCV uses the fontawesome and academicon fonts
%% and packages.
%% See texdoc.net/pkg/fontawecome and http://texdoc.net/pkg/academicons for full list of symbols. You MUST compile with XeLaTeX or LuaLaTeX if you want to use academicons.

% Change the page layout if you need to
\geometry{left=2cm,right=10cm,marginparwidth=6.8cm,marginparsep=1.2cm,top=2cm,bottom=1.25cm}

% Change the font if you want to, depending on whether
% you're using pdflatex or xelatex/lualatex
\ifxetexorluatex
  % If using xelatex or lualatex:
  \setmainfont{Carlito}
\else
  % If using pdflatex:
  \usepackage[utf8]{inputenc}
  \usepackage[T1]{fontenc}
  \usepackage[default]{lato}
\fi

% Change the colours if you want to
\definecolor{VividPurple}{HTML}{000000}
\definecolor{SlateGrey}{HTML}{2E2E2E}
\definecolor{LightGrey}{HTML}{2E2E2E}
\colorlet{heading}{VividPurple}
\colorlet{accent}{VividPurple}
\colorlet{emphasis}{SlateGrey}
\colorlet{body}{LightGrey}

% Change the bullets for itemize and rating marker
% for \cvskill if you want to
\renewcommand{\itemmarker}{{\small\textbullet}}
\renewcommand{\ratingmarker}{\faCircle}

%% sample.bib contains your publications
% \addbibresource{sample.bib}

\begin{document}
\name{Alfredo Alarcon Y.}
\tagline{Scientific Informatics}
% Cropped to square from https://en.wikipedia.org/wiki/Marissa_Mayer#/media/File:Marissa_Mayer_May_2014_(cropped).jpg, CC-BY 2.0
%\photo{3.3cm}{profile.jpg}
\personalinfo{%
  % Not all of these are required!
  % You can add your own with \printinfo{symbol}{detail}
  \email{a.alarcon.y@gmail.com}
  \phone{+33 6 86 16 56 09}
  %  \mailaddress{Address, Street, 00000 County}
  \location{Grenoble, France}
  \printinfo{Age}{34 years old}
  \printinfo{Nationality}{Chilean / French}
  %  \homepage{alfredoalarcon.com}
  %  \twitter{@marissamayer}
  %  \linkedin{linkedin.com/in/ronak-dedhiya}
  %  \github{github.com/ronakdedhiya} % I'm just making this up though.
  %   \orcid{orcid.org/0000-0000-0000-0000} % Obviously making this up too. If you want to use this field (and also other academicons symbols), add "academicons" option to \documentclass{altacv}
}

%% Make the header extend all the way to the right, if you want.
\begin{fullwidth}
  \makecvheader
\end{fullwidth}

%% Depending on your tastes, you may want to make fonts of itemize environments slightly smaller
\AtBeginEnvironment{itemize}{\small}

%% Provide the file name containing the sidebar contents as an optional parameter to \cvsection.
%% You can always just use \marginpar{...} if you do
%% not need to align the top of the contents to any
%% \cvsection title in the "main" bar.


\cvsection[page1sidebar]{DESCRIPTION}

\small{Scientific Engineer \textbf{with 8-year experience in Data Science and  Web Development}. Interested in innovation, multidisciplinarity and continuous learning.
Please visit my website for more information about my professional background - \textbf{\href{http://alfredoalarcon.com}{http://alfredoalarcon.com}}.

\cvsection{Experience}

\cvevent{Bioinformatics Team Leader}{\href{https://www.global-bioenergies.com/?lang=en}{Global Bioenergies}}{Feb 2017 -- Sept 2019}{Paris region, France}
\begin{itemize}
  \item I built a new activity in bioinformatics, data science and web development in a \textbf{biotechnology company}.
  \item Development and deployment of a whole web application useful for processing, analysis and visualization of most scientific data of the company.
  \item Modelling and analysis of \textbf{genomics and transcriptomics data} led to the discovery of new genes and metabolic pathways. Three patents being written at the moment.
  \item Management and training of one collaborator.
  \item Dissemination of data analysis and web development among biologist colleagues.
  \item Tools based on python, javascript (React), SQL (postgres) and devops, as well as specialised bioinformatics software (clustalo, rdkit, biopython, blast, hmmer...).
\end{itemize}

\divider

\cvevent{Transportation Data Analyst}{\href{https://www.systra.com/en/}{Systra}}{ Jan 2013 -- July 2016 }{Marseille, France }
\begin{itemize}
  \item Public policies analysis applied to \textbf{public transportation planning}.
  \item Analysis of geographical, transportation and poll data in order to forecast demand for a public transportation system in a regional or national context (\textit{Transportation networks analysis}).
  \item Data mining and forecast using python, excel and VBA, as well as specialised transportation software.
\end{itemize}

\divider


\vspace{1.5em}

\cvsection{Education}
\cvevent{Engineer - \small{\textit{Ing\'enieur de l'Ecole Polytechnique}}}
  {\href{https://www.polytechnique.edu/en}{Ecole Polytechnique} (X2008)}
  {December 2012}
  {Palaiseau, France}
\vspace{0.5em}

\cvevent{MSc Biotechnology - \small{\textit{Systems and Synthetic Biology}}}
  {\href{https://www.universite-paris-saclay.fr/en}{Universit\'e Paris-Saclay}}
  {January 2016}
  {Evry, France}
\vspace{0.5em}

\cvevent{MSc Economics and Public Policies}{\href{https://www.sciencespo.fr/en}{Sciences Po} - \href{https://www.polytechnique.edu/en}{Ecole Polytechnique}}{December 2012}{Paris, France}
%\divider

% \divider






\clearpage

% \cvsection[page2sidebar]{Publications}

\nocite{*}

% \printbibliography[heading=pubtype,title={\printinfo{\faBook}{Books}},type=book]

% \divider

% \printbibliography[heading=pubtype,title={\printinfo{\faFileTextO}{Journal Articles}}, type=article]

% \divider

% \printbibliography[heading=pubtype,title={\printinfo{\faGroup}{Conference Proceedings}},type=inproceedings]

% %% If the NEXT page doesn't start with a \cvsection but you'd
% %% still like to add a sidebar, then use this command on THIS
% %% page to add it. The optional argument lets you pull up the
% %% sidebar a bit so that it looks aligned with the top of the
% %% main column.
% % \addnextpagesidebar[-1ex]{page3sidebar}


\end{document}
